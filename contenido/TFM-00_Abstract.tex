\begin{abstract}[Resumen]
En la monitorización de redes informáticas a gran escala, resulta de alto interés conocer el comportamiento de sus equipos finales y detectar aquellos que puedan ser sospechosos.
Sin embargo, clasificar a cada equipo según su actividad supone un importante ejercicio de síntesis.
Además, es difícil obtener unas categorías útiles, sobre todo en el caso de las anomalías, ya que son desconocidas a priori.
El presente trabajo aborda este reto haciendo uso de técnicas de \emph{clustering} a partir de logs extraídos de firewalls.
Con un muestreo del 5 \%, que supone un millón de sesiones al día, se han distinguido 5 clases de comportamientos.
Los comportamientos anómalos se han conseguido reunir en un solo cluster con menos de diez casos al día.
\par\vspace{0.25cm}
\centering\textbf{Palabras clave:} \keywordsESv \par
\end{abstract}


\pagebreak
%Ingles
\begin{abstract}
In network monitoring at a large scale, knowing how the hosts behave and detecting those who might be suspicious is of high interest.
Nevertheless, classifying each host according to its activity requires a major summary exercise.
Besides, it is hard to get useful categories, especially in the case of anomalies, since they are a priori unknown.
The present work addresses this challenge using clustering techniques on firewall logs.
With a sampling of 5 \%, which means a million of sessions each day, 5 classes have been identified.
The anomalous behaviours have been gathered in a single cluster with less than ten cases each day.
\par\vspace{0.25cm}
\centering\textbf{Keywords:} \keywordsv \par
\end{abstract}
